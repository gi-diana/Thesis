% !TEX encoding = UTF-8 Unicode
% !TEX TS-program = pdflatex
%
% Esempi d'uso degli ambienti frontespizio (e frontespizio*)
\documentclass[%
% copo=10pt,10pt
corpo=11pt,
% corpo=12pt,
twoside,
 stile=classica,
oldstyle,
%autoretitolo,
greek,% per comporre parte del testo in greco (qui solo con classica)
%cucitura,
]{toptesi}
% \usepackage{teubner}% per le notazioni filologiche greche
%
%%%%%%%%%%%%%%%%%%%%%%%%%%%%%%%%%%%%%%%%%%%%%%%%%%%%%%%%%%%%
% per macchine Linux/Mac/UNIX/Windows; sarebbe meglio utf8
%\usepackage[latin1]{inputenc}
\usepackage[utf8]{inputenc}
\usepackage[T1]{fontenc}
\usepackage{lmodern}
\usepackage{textcomp}

%%%%%%%%%%%%%%%%%%%%%%%%%%%%%%%%%%%%%%%%%%%%%%%%%%%%%%%%%%%%%%%%%
\usepackage{lipsum}% Per scrivere testo fasullo in "latinorum"

\usepackage{pict2e}% elimina ogni limitazione all'ambiente picture
%

%\english%  di default la lingua è impostata con \italiano


\iflanguage{english}{%
	\retrofrontespizio{This work is subject to the Creative Commons Licence}
	\DottoratoIn{PhD Course in\space}
	\CorsoDiLaureaIn{Master degree course in\space}
	\NomeMonografia{Bachelor Degree Final Work}
	\TesiDiLaurea{Master Degree Thesis}
	\NomeDissertazione{PhD Dissertation}
	\InName{in}
	\CandidateName{Candidates}% or Candidate
	\AdvisorName{Supervisors}% or Supervisor
	\TutorName{Tutor}
	\NomeTutoreAziendale{Internship Tutor}
	\CycleName{cycle}
	\NomePrimoTomo{First volume}
	\NomeSecondoTomo{Second Volume}
	\NomeTerzoTomo{Third Volume}
	\NomeQuartoTomo{Fourth Volume}
	\logosede{logouno}% or comma separated list of logos
}{}
%%%%%%%%%%%%%%%%%%%%%%%%%%%%%%%%%%%%%%%%%%%%%%%%%%%%%%%%%%%%%%%%%%%%%%%%%%%%%%%%%%%%%%%%

%%%%%%%%%%%%%%%%%%%%%%%%%%%%%%%%%%%%%%%%%%%%%%%%%%%%%%%%%%%%%%%%%%%%%%%%%%%
% Lasciare questo per ultimo dopo aver caricato tutti gli altri pacchetti
%
% Commentare la riga seguente se si è specificata l'opzione "pdfa"
\usepackage{hyperref}

\hypersetup{%
    pdfpagemode={UseOutlines},
    bookmarksopen,
    pdfstartview={FitH},
    colorlinks,
    linkcolor={blue},
    citecolor={blue},
    urlcolor={blue}
  }
%%%%%%%%%%%%%%%%%%%%%%%%%%%%%%%%%%%%%%%%%%%%%%%%%%%%%%%%%%%%%%%%%%%%%%%%%

%%%%%%% Definizioni locali
\newtheorem{osservazione}{Osservazione}% Standard LaTeX



\begin{document}
\errorcontextlines=9% debugging


%%%%%%%%%%%%%%%%%%%%%%%%%%%%%%%%%%%%%%%%%%%%%%%%%%%%%%%%%%%%%%%%%%%%%%%%%%%%%%%%%%%%%%
% Questo codice serve per collaudare gli ambienti frontespizio e frontespizio*
% con l'asterisco il logo dell'ateneo va a finire sotto il titolo nella seconda
% metà della pagina; senza asterisco va a finire in testa.

\begin{frontespizio}
\ateneo{Politecnico di Torino}% nome generico dell'universita'
%\ateneo{}% nome generico vuoto per sperimentare nei vari casi
\nomeateneo{}% nome proprio dell'universita'
\FacoltaDi{}% prefisso vuoto per la facolta'
\facolta[III]{}% dottrine della facolta'
\titolo{Dimensionamento di un braccio robotico a 6 assi}% per la laurea quinquennale e il dottorato
\sottotitolo{Progetto rover Trinity - Team DIANA}% per la laurea quinquennale e il dottorato
\corsodilaurea{Ingegneria Meccanica}% per la laurea
\renewcommand*\IDlabel{\\\quad matricola: }% per ridefinire il prefisso del numero di matricola
\candidato{Luigi \textsc{Di Rado}}[204427]% per tutti i percorsi
%\secondocandidato{Evangelista \textsc{Torricelli}}[123457]% solo per l. magistrale 
\relatore{prof.\ Stefano Pastorelli}% per la laurea e/o il dottorato
%\secondorelatore{dipl.~ing.~Werner von Braun}% solo per la laurea magistrale
%\tutoreaziendale{dott.\ ing.\ Giovanni Giacosa}
%\NomeTutoreAziendale{Supervisore aziendale\\Centro Ricerche FIAT}
\sedutadilaurea{\textsc{Anno~accademico} 2019-2020}% per la laurea magistrale
\ciclodidottorato{XV}% solo per il dottorato
\logosede{logouno} % questo e' ovviamente facoltativo
\end{frontespizio}
%%%%% 


% \paginavuota % funziona anche senza specificare l'opzione classica

\ringraziamenti % stampa i ringraziamenti, di solito superflui e inutili

I candidati ringraziano vivamente il Granduca di Toscana per i mezzi
messi loro a disposizione, ed il signor Von Braun, assistente del
prof.~Albert Einstein, per le informazioni riservate che egli ha
gentilmente fornito loro, e per le utili discussioni che hanno permesso
ai candidati di evitare di riscoprire l'acqua calda.



\mainmatter

\chapter{Introduzione generale}
\section{Rover}


\begin{thebibliography}{9}
\bibitem{gal} G.~Galilei, {\em Nuovi studii sugli astri medicei}, Manuzio,
        Venetia, 1612.
\bibitem{tor1} E.~Torricelli, in ``La pressione barometrica'', {\em Strumenti
        Moderni}, Il Porcellino, Firenze, 1606.
\bibitem{tor2} E.~Torricelli e A.~Vasari, in ``Delle misure'', {\em Atti Nuovo
        Cimento}, vol.~III, n.~2 (feb. 1607), p.~27--31.
\bibitem{duane1964} Duane J.T., \emph{Learning Curve Approach To Reliability 
		Monitoring}, IEEE Transactions on Aerospace, Vol. 2, pp. 563-566, 1964
\end{thebibliography}

\end{document}

% altri riferimenti da usare come esempi.

\bibitem{chiesa2008} Chiesa S., \emph{Affidabilità, sicurezza e manutenzione 
		nel progetto dei sistemi}, CLUT, gennaio 2008
\bibitem{chiesa2}Chiesa S., Fioriti M., Fusaro R., \emph{On Board System 
		Technological  Level Improvement Effect on UAV MALE}
\bibitem{bigliano2010} Bigliano M., \emph{Sicurezza nell'installazione di un velivolo 
		senza pilota MALE; applicazione di metodologia di Zonal Safety 
		Analysis al velivolo del Progetto SAvE}, Politecnico di Torino, 
		maggio 2010
\bibitem{astrid2012} Chiesa S., Di Meo G.A., Fioriti M., Medici G., Viola N.,
		\emph{ASTRID - Aircraft on board Systems sizing and TRade-off 
		analysis in Initial Design}, Research Bulletin, Warsaw University 
		of Technology, Institute of Aeronautics and Applied Mechanics, 
		p. 1-28, 17-19, ottobre 2012

